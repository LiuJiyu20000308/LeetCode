\documentclass[a4paper,twoside]{ctexart}
\usepackage{geometry}
\geometry{margin=1cm,vmargin={0pt,1cm}}
\setlength{\topmargin}{-2cm}
\setlength{\paperheight}{23cm}
\setlength{\paperwidth}{18cm}
\setlength{\textheight}{19.6cm}
\setlength{\textwidth}{15cm}
\usepackage{makecell}
%\usepackage{fancyhdr}
\usepackage{siunitx}
\usepackage{amssymb}
\usepackage{indentfirst}
\setlength{\parindent}{0.5em}

\pagenumbering{arabic} 

% useful packages.
\usepackage{multirow}
\usepackage{caption}
\usepackage{mathrsfs}
\usepackage{amsfonts}
\usepackage{amsmath}
\usepackage{amsthm}
\usepackage{enumerate}
\usepackage{xcolor,graphicx,float,subfigure}
\usepackage{epstopdf}
\usepackage{multicol}
\usepackage{fancyhdr}
\usepackage{layout}
\usepackage{listings}
\usepackage{dsfont}
\lstset{language=Matlab}
\lstset{breaklines}
\lstset{extendedchars=false}
\usepackage[colorlinks,linkcolor=blue]{hyperref}
\usepackage{xcolor}
%\usepackage{cite}
%\usepackage[numbers,sort&compress]{natbib} 
%\setcitestyle{open={},close={}}
%\usepackage{natbibspacing}
%\renewcommand{\refname}{}
\usepackage{anyfontsize}

\usepackage{tikz}
\usetikzlibrary{calc}
\usetikzlibrary{arrows.meta}
\tikzset{
  dot/.style={
    circle, fill=black, inner sep=1pt, outer sep=0pt
  },
  dot label/.style={
    circle, inner sep=0pt, outer sep=1pt
  }
  arrow1/.style = {
    draw = black, thick, -{Latex[length = 4mm, width = 1.5mm]},
  }
}

\newtheorem{theorem}{定理}[section]
\newtheorem{corollary}[theorem]{推论}
\newtheorem{lemma}[theorem]{引理}
\newtheorem{definition}[theorem]{定义}
\newtheorem{proposition}[theorem]{性质}
\newtheorem{example}[theorem]{例子}
\newtheorem{notation}[theorem]{记号}
\newtheorem{algorithm}[theorem]{算法}


\newcommand{\dif}{\mathrm{d}}
\newcommand{\avg}[1]{\left\langle #1 \right\rangle}
\newcommand{\difFrac}[2]{\frac{\dif #1}{\dif #2}}
\newcommand{\pdfFrac}[2]{\frac{\partial #1}{\partial #2}}
\newcommand{\OFL}{\mathrm{OFL}}
\newcommand{\UFL}{\mathrm{UFL}}
\newcommand{\fl}{\mathrm{fl}}
\newcommand{\op}{\odot}
\newcommand{\Eabs}{E_{\mathrm{abs}}}
\newcommand{\Erel}{E_{\mathrm{rel}}}

\newcommand{\Zero}{\hat{0}}
\newcommand{\One}{\hat{1}}
\newcommand{\Int}{\mathrm{int}}
\newcommand{\unitV}{\mathds{1}}

\newcommand{\bmi}{\mathbf{i}}
\newcommand{\bmj}{\mathbf{j}}
\newcommand{\bmn}{\mathbf{n}}

\newcommand{\dist}[2]{\text{dist}\left(#1, #2\right)}
\newcommand{\scientific}[2]{#1 \times 10^{#2}}


%\newcommand{\Dim}{{\mathbf{D}}}
\newcommand{\Dim}{{\scriptsize \textsf{D}}}
\newcommand{\me}{\mathrm{e}}
\newcommand{\mi}{\mathrm{i}}

%\newcommand{\mod}{\mathrm{mod}}
\newcommand{\curve}[1]{\widetilde{#1}}
%\newcommand{\dt}{\delta t}
\newcommand{\dt}{\tau}
\newcommand{\isCovered}{\mathbin{ < \! \! \! \! \cdot }}
%\newcommand{\cIncluded}{\mathbin{ \prec \! \! \! \cdot }}
\newcommand{\coveredBy}{\lhd}
%\newcommand{\regrz}[1]{\mathrm{cl}\left(\mathrm{int}\left(#1\right)\right)}
\newcommand{\regrz}[1]{\mathrm{reg}\left(#1\right)}
%\newcommand{\sgncup}{\ \hat{\cup} \ }
\newcommand{\Span}{\mathrm{span}}
\newcommand{\timeline}[2]{\phi_{t_0}^{#1}\left( #2 \right)}
\newcommand{\timeBP}[1]{\overleftarrow{#1}}
\newcommand{\timeBPA}[1]{\mathring{\overleftarrow{#1}}}
\newcommand{\streak}[2]{\Psi_{t_0}^{#1}\left(#2\right)}
\newcommand{\timelineA}[2]{\mathring{\phi}_{t_0,#2}^{#1}}
\newcommand{\DRLN}[1]{{\cal D}_{\curve{#1}}}
\newcommand{\DRLLN}[1]{{\cal D}_{\overline{#1}}}
\newcommand{\DRLNA}[1]{\mathring{\cal D}_{\curve{#1}}}
%\newcommand{\oplusDR}{\,\overline{\oplus}\,}
\newcommand{\oplusDR}{\,\bar{\oplus}\,}
\newcommand{\qo}{\hat{q}}
\newcommand{\xo}{\hat{x}}
\newcommand{\yo}{\hat{y}}
\newcommand{\closure}[1]{\textrm{cl}\left(#1\right)}
\newcommand{\vertexSequence}[4]{
  \left( #1 \rightarrow #2 \rightarrow #3 \rightarrow #4 \rightarrow #1\right)}

\newcommand{\ppSpace}{\Pi_{<\kappa,\bm{\xi},\bm{\nu}}}
\newcommand{\pnSpace}{\mathbb{P}_{<\kappa}}
\newcommand{\pnSpaceK}[1]{\mathbb{P}_{#1}}

\newcommand{\Pyr}[2]{\textrm{Pyr}_{\cal{#1}}\left(\mathbf{#2}\right)}

%\pagestyle{plain}
\pagestyle{fancy}
\fancyhf{}
\fancyhead[LE,RO]{\textbf{\thepage}}

\makeatletter
\newcommand\sixteen{\@setfontsize\sixteen{17pt}{6}}
\renewcommand{\maketitle}{\bgroup\setlength{\parindent}{0pt}
\begin{flushleft}
\sixteen\bfseries \@title
\medskip
\end{flushleft}
\textit{\@author}
\egroup}
\makeatother

\CTEXsetup[format={\Large\bfseries}]{section}

\title{刷题算法总结}


\begin{document}
\maketitle

\section{查找}
\label{sec:Search}

\subsection{二分查找}
\label{sec:BinarySearch}
\textbf{对应题目:}
\begin{itemize}
\item 33. 搜索\textbf{旋转}排序数组
\item \textbf{34. 在排序数组中查找元素第一个和最后一个出现位置}
\item 74. 搜索二维矩阵
\item 153. 寻找旋转排序数组中的最小值
\item \textbf{162. 寻找峰值 (涉及Lambda函数技巧)}
\end{itemize}

涉及到的标准库: binary\_search(),upper\_bound (大于),以及 lower\_bound(不小于)

注意细节,ans永远取有可能是最后结果的那个,举个例子

\begin{verbatim}
int binarySearch(vector<int>& nums, int target, bool lower) {
        int left = 0, right = (int)nums.size() - 1, ans = (int)nums.size();
        while (left <= right) {
            int mid = (left + right) / 2;
            if (nums[mid] > target || (lower && nums[mid] >= target)) {
                right = mid - 1;
                ans = mid;
            } else {
                left = mid + 1;
            }
        }
        return ans;
    }
\end{verbatim}

另外,不一定有序数组才可以使用二分算法,例如寻找数组旋转点、寻找
峰值。


\section{双指针}

双指针是指在遍历对象时,使用两个或多个指针进行遍历及相应的操作。大多用
于数组操作,这利用了数组连序性的特点。双指针常用来降低算法的时间复杂度,
因为使用两个指针可以避免多层循环。通常有对撞指针和快慢指针。

\textbf{对应题目:}
\begin{itemize}
\item \textbf{82.删除排序链表重复元素}
\item 15.三数之和
\item \textbf{844. 比较含退格字符}
\item 986.区间列表交集
\item \textbf{11.盛最多水的容器}
\end{itemize}

双指针的重点在于如何设置双指针,指针每一步该如何移动,包括方向和速度。

\subsection{滑动窗口}

滑动窗口就是暴力解法的优化,是在给定特定大小的字符串或者数组上操作,而
不在整个数组上操作。进而减少了问题的复杂度和嵌套深度。

\textbf{对应题目:}
\begin{itemize}
\item \textbf{438. 找到字符串中所有异位子字符}
\item \textbf{713. 乘积小于k的子数组}
\item 209. 长度最小的子数组
\end{itemize}

\section{BFS/DFS}

\subsection{岛屿网格类问题}

可以与二叉树的DFS进行类比,可以通过额外数组进行辅助,例如存储是否已经
被遍历过、改变网格值表示是否为同一个岛屿等等。

\textbf{对应题目:}
\begin{itemize}
\item 200.岛屿数量
\item 547.省份数量
\item 827.填海造陆
\item \textbf{1091. 二进制矩阵中的最短路径} 注意当把它加入到queue的时
  候就直接visited=1。
\item \textbf{130.被围绕的区域}  注意dfs的时候首先判断要不要跳过它,跳
  过要注意齐全,是否满足条件,是否越界,是否以及visited。
\end{itemize}

\subsection{二叉树上的操作}

BFS的话使用queue,需要逐层操作的时候while内部要嵌套一个for循环

对应题目:
\begin{itemize}
\item 117.填充每个节点的下一个右侧节点指针
\item 572.另一棵树的子树(有高端解法)(KMP算法对比字符串)
\end{itemize}

\section{递归/回溯}

\subsection{回溯}

回溯的步骤:
\begin{itemize}
\item 寻找出递归树以及结束条件
\item 找选择列表
\item 判断是否剪枝
\item 做出选择
\item 撤销选择
\end{itemize}

\begin{verbatim}
void backtracking(参数) {
    if (终止条件) {
        存放结果;
        return;
    }

    for (选择:本层集合中元素(树中节点孩子的数量就是集合的大小)) {
        处理节点;
        backtracking(路径,选择列表); // 递归
        回溯,撤销处理结果
    }
}

\end{verbatim}

有时候也要用到visited避免重复搜索,注意遇到重复的时候下一步该做什么,
而不是直接return。 另外,回溯时只有遍历到最后或者条件达成的
时候才将结果放入到vector中。

\textbf{对应题目:}
\begin{itemize}
\item \textbf{797. 所有可能的路径}
\item 78. 子集
\item 90. 子集II(包含重复元素)
\item \textbf{46. 全排列}
\item \textbf{47. 全排列II}
\item 39.组合求和
\item \textbf{40.组合求和II}
\item 17. 电话号码
\item 22. 括号生成
\item 79. 单词搜索
\end{itemize}


\section{动态规划}

动态规划分为以下几步:
\begin{enumerate}
\item 确定dp数组及其下标的含义
\item 确定递推公式
\item dp数组如何初始化
\item 确定遍历顺序
\item 举例推导dp数组
\end{enumerate}

\subsection{基础题目}
着重注意使用滚动数组技巧。

\begin{itemize}
\item 509. 斐波那契数列
\item \textbf{70. 爬楼梯}
\item 746. 最小花费爬楼梯
\item 62. 不同路径,63.不同路径II
\item 343. 整数拆分
\item \textbf{96. 不同的二叉搜索树}
\end{itemize}

\subsection{01背包问题}
每个物品只能放一次,始终注意如何将一个问题转化为01背包问题,滚动数组要
倒序遍历。

\begin{itemize}
\item \item \textbf{416. 分割等和}
\item \textbf{1049. 最后一块石头}
\item \textbf{494. 目标和}
\item \textbf{474. 一和零}
\end{itemize}

\subsection{完全背包问题}
每个物品可以放无数次,滚动数组要正序遍历。\textbf{着重注意初始化以及递
  推!!!}

关于方法个数的问题,先遍历背包是排列数,先遍历物品是组合数。

\begin{itemize}
\item \textbf{518. 零钱兑换II}
\item \textbf{377. 组合总和IV}
\item \textbf{70. 爬楼梯 加入一次可以爬1-m个台阶呢? 排列数问题}
\item 322. 零钱兑换
\item \textbf{139.单词拆分} 也可以用回溯算法
\item 279. 完全平方数
\end{itemize}


\subsection{打家劫舍}

\begin{itemize}
\item 198. 打家劫舍
\item 213.打家劫舍II 
\item 337.打家劫舍III  \textbf{树状dp} dp返回两个数,一个是不偷该节点
  的最大利润,一个是偷该节点的最大利润
\end{itemize}

\subsection{股票}

每天有多种状态,每一天每种状态所获得的最大利润都要记录下来

\begin{itemize}
\item \textbf{121. 买卖股票最佳时机, }   只能购买一次
\item \textbf{122. 买卖股票最佳时机II,} 可以无数次购买
\item \textbf{123. 买卖股票最佳时机III,} 最多购买两次(4种状态)
\item \textbf{188. 买卖股票最佳时机IV,} 最多购买k次(2k种状态)
\item \textbf{309. 买卖股票最佳时机含冰冻期,} 可以无数次购买但有一天冰
  冻期(4种状态:持有股票、未持有股票但已经过了冰冻期、刚卖、冰冻期)
\item \textbf{714.买卖股票最佳时机含手续费,} 和122很像

\end{itemize}

\subsection{子序列问题}

\subsubsection{不连续子序列}

注意dp的下标含义,思考为什么有时候需要记录以i结尾的子序列有时候不需要

\begin{itemize}
\item 最长上升子序列
\item 最长公共子序列
\item 不相交的线
\end{itemize}

\subsubsection{连续子序列}

注意dp的下标是以i结尾的满足要求的子序列长度

\begin{itemize}
\item 674. 最长连续递增序列
\item 718. 最长重复子数组
\item 53. 最大子序和
\end{itemize}

\subsection{子序列问题}

注意dp的下标含义

\begin{itemize}
\item 392. 判断子序列
\item \textbf{115. 不同的子序列}
\item 583. 两个字符串的删除操作
\item \textbf{72. 编辑距离}
\end{itemize}

\subsection{回文}

dp[i][j]表示[i,j]范围是否是回文或者长度
\begin{itemize}
\item \textbf{647. 回文子串}
\item \textbf{516. 最长回文子序列长度}
\end{itemize}

\end{document}
%%% Local Variables:
%%% mode: latex
%%% TeX-master: t
%%% End:
